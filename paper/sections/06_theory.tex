\section{Theory: Permutation-Coupled Localization}
\label{sec:theory}

We build our theoretical understanding of \sketchfamily on the framework of localized sketching developed in~\cite{srinivasa2020localized}.
\emph{Localized sketching} studies block-diagonal (or block-local) sketches whose guarantees depend on how the target subspace distributes across the chosen blocks.
Our sketch \sketchfamily follows a very similar locality principle with the key difference that a single output block mixes information from multiple input blocks via the union-of-permutations wiring.
We formalize this via a new \emph{neighborhood coherence} quantity that is a natural generalization of the block coherence in~\cite{srinivasa2020localized}.

\subsection{Coherence for Permutation-Coupled Localization}
\label{sec:theory_coherence}
We begin by defining our generalization of block-coherence~\Cref{def:block_coherence}, the \emph{neighborhood coherence}.

\begin{definition}[Neighborhood coherence for permutation wiring]
  \label{def:neighborhood_coherence}
  Fix $\kappa$ \textbf{edge-disjoint} permutations $\pi_1,\ldots,\pi_\kappa$ on $[M]$ and define the neighborhood of output block $g\in[M]$ by
  \begin{equation}
    \label{eq:nbr_def}
    \mathcal{N}(g) \;:=\; \{\pi_\ell(g)\}_{\ell=1}^\kappa.
  \end{equation}
  Let $U_{\mathcal{N}(g)}\in\R^{(\kappa\Bc)\times r}$ denote the matrix obtained by stacking the blocks
  $U^{(h)}$ for $h\in\mathcal{N}(g)$.
  The neighborhood coherence of $U$ under $\pi$ is
  \begin{equation}
    \label{eq:mu_nbr_def}
    \mu_{\mathrm{nbr}}(U;\pi)
    \;:=\;
    \frac{M}{\kappa}\max_{g\in[M]}\,\|U_{\mathcal{N}(g)}\|_2^2.
  \end{equation}
\end{definition}
Recall that the edge-disjoint condition ensures that each output block $g$ connects to $\kappa$ distinct input blocks.

\subsection{Oblivious Subspace Embedding Guarantee}
\label{sec:theory_ose}

We now state our main theoretical result \Cref{thm:ose_informal}, which is an OSE guarantee for \sketchfamily controlled by neighborhood coherence.
The flavor is very similar to the localized SJLT result of~\cite{srinivasa2020localized}, but with block coherence replaced by neighborhood coherence.

\begin{theorem}[OSE for \sketchfamily]
  \label{thm:ose_informal}
  Fix $U\in\R^{d\times r}$ with orthonormal columns.
  Let $S\sim\sketchfamily$ with parameters $(M,\Br,\kappa,s)$ and embedding dimension $k=M\Br$.
  Let $t := r+\log\frac{1}{\delta}$.
  There exist absolute constants $C,c>0$ such that, if we have
  \begin{equation}
    \label{eq:ose_informal_cond}
    k \;\ge\; C\,\frac{\mu_{\mathrm{nbr}}(U;\pi)}{\varepsilon^2}\,t
    \qquad\text{and}\qquad
    \kappa s \;\ge\; C\,\frac{1}{\varepsilon}\,t,
  \end{equation}
  then with probability at least $1-\delta$ over $S$,
  \begin{equation}
    \label{eq:ose_informal_concl}
    \bigl\|U^\top S^\top S U - I_r\bigr\|_2 \le \varepsilon.
  \end{equation}
\end{theorem}
The proof follows the localized sketching blueprint closely.
For a fixed vector $w$, we decompose $\|SUw\|_2^2$ into a sum of independent block contributions and combine a fixed-vector tail bound with a net argument.
The union-of-permutations wiring enters through a simple energy identity that ensures the local neighborhoods collectively cover the input without bias.
A detailed proof appears in \Cref{sec:app_theory}.
Note that for $\kappa=1$, \Cref{thm:ose_informal} recovers the localized SJLT regime controlled by $\mu_{\mathrm{blk}}(U)$ from~\cite{srinivasa2020localized}.

\subsection{Controlling Neighborhood Coherence with Randomized Permutations}
We now discuss how neighborhood coherence interpolates between block coherence and fully mixed coherence, and how random permutations can improve it in practice.

\paragraph{Worst-case comparison.}
First, a simple worst case comparison between neighborhood coherence and block coherence shows that permutations can reduce coherence by up to a factor of $\kappa$, but in general we cannot do better than block coherence.
For any fixed set of edge-disjoint permutations $\pi$,
\begin{equation}
  \label{eq:mu_nbr_bounds}
  \frac{1}{\kappa}\,\mu_{\mathrm{blk}}(U)
  \;\le\;
  \mu_{\mathrm{nbr}}(U;\pi)
  \;\le\;
  \mu_{\mathrm{blk}}(U).
\end{equation}
The upper bound is a triangle inequality and the lower bound follows because every block appears in at least one neighborhood.
A detailed derivation appears in \Cref{sec:app_theory}.

\paragraph{Randomized Permutations.}
Now, if we model the permutations $\pi_1,\ldots,\pi_\kappa$ as independent and uniformly random, then we can show a stronger upper bound that improves with $\kappa$ and pushes $\mu_{\mathrm{nbr}}(U;\pi)$ toward one.
A precise statement appears as \Cref{prop:perm_smoothing} in \Cref{sec:app_theory}.
Informally, we have the following smoothing bound.
Let $L := \log\!
  \bigl(\tfrac{Mr}{\delta}\bigr)$. With probability at least $1-\delta$ over the permutations, we have
\begin{equation}
  \label{eq:perm_smoothing_main}
  \mu_{\mathrm{nbr}}(U;\pi)
  \;\le\;
  1 + C\left(
  \sqrt{\frac{\mu_{\mathrm{blk}}(U)\,L}{\kappa}}
  + \frac{\mu_{\mathrm{blk}}(U)\,L}{\kappa}
  \right)
\end{equation}
for an absolute constant $C>0$.

This rigorously justifies the benefit of using multiple permutations to improve mixing.
When $\kappa$ is large enough, the neighborhood coherence approaches one, which is the optimal coherence for any $U$, improving the OSE guarantee in \Cref{thm:ose_informal}.

\begin{remark}[Independent permutations in practice]
  Our analysis models $\pi_1,\ldots,\pi_\kappa$ as independent uniformly random permutations.
  In \method, we generate $\kappa$ distinct permutations using a lightweight family for efficiency, see \Cref{sec:app_perm_wiring}.
  When $\kappa\ll M$, sampling distinct permutations is close to sampling independently, and collisions are rare.
  This justifies our theoretical model.
\end{remark}
