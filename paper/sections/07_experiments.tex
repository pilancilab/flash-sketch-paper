
\section{Experiments}
\label{sec:experiments}

We empirically evaluate \method at two levels.
First, we treat sketching as a primitive and measure the runtime and distortion of computing $SA$ for dense $A$ across a broad range of shapes.
Second, we integrate \method into end-to-end workloads where sketching sits on the critical path.
As part of this, we consider a suite of standard RandNLA tasks as well as GraSS, an end-to-end ML application for data attribution~\cite{hu2025grass} in which sketching is a core component.
We time using CUDA events and report the mean over 10 iterations (after warm-up iterations).
We restrict our evaluation to FP32 arithmetic for both sketching and downstream tasks for consistency, since all our baselines support FP32.
Our main kernel idea extends naturally to other precisions as well, but we leave concrete exploration of precision-specific optimizations to future work.

\subsection{Baselines}
We compare against dense and sparse GPU baselines.
\begin{enumerate}
  \item A dense JL transform with a standard Gaussian projection using cuBLAS.
  \item A sparse JL transform using cuSPARSE SpMM.
  \item A sparse JL transform using the GraSS SJLT kernel~\cite{hu2025grass}.
  \item A subsampled randomized Hadamard transform (SRHT) using the Fast Hadamard Transform (FHT) kernel from Dao-AILab~\cite{daoailab_fast_hadamard_transform}.
\end{enumerate}

\subsection{Sketching}
For the primitive sketching evaluation, we use the relative error in the computed Gram matrix $\|A^\top A-(SA)^\top(SA)\|_F/\|A^\top A\|_F$ as our quality metric.
The preview in \Cref{fig:pareto_money} illustrates the quality--speed tradeoff on a representative input from GPT2-medium weights.
More extensive ablations and Pareto frontiers across different shapes and inputs appear in \Cref{sec:app_additional_experiments}.

\subsection{RandNLA tasks}
We benchmark \method on three standard end-to-end RandNLA tasks:
\begin{enumerate}
  \item Sketch-and-solve least squares regression.
  \item Sketch-and-ridge regression.
  \item Oblivious subspace embedding (OSE).
\end{enumerate}
We run each task on a variety of input shapes and types.
We use the following datasets for inputs:
\begin{enumerate}
  \item A synthetic Gaussian matrix.
  \item A synthetic low-rank + noise matrix.
  \item A sparse submatrix from the SuiteSparse collection (\Verb|spal_004|, with nonzero density $\approx 1.4\%$)~\cite{kolodziej2019suitesparse}.
  \item A concatenated subset of weights from large language models (LLMs) (\Verb|GPT2-medium|~\cite{radford2019language} and \Verb|Qwen2-1.5B|~\cite{yang2024qwen2}).
\end{enumerate}
\Cref{fig:sketch_and_ridge_pareto} shows a representative quality--speed Pareto frontier for sketch-and-ridge regression on Qwen2-1.5B stacked weights.
\begin{figure}[h]
  \centering
  \includegraphics[width=\linewidth]{figures/fig_e2e_camera_ready_ridge_regression_qwen_qwen2_1_5b_qwen2_1p5b_weights_full_64k_512_residual_nvidia_geforce_rtx_4090.pdf}
  \caption{\textbf{Sketch-and-ridge regression on Qwen2-1.5B stacked weights.}
    We compare end-to-end runtime on an NVIDIA RTX 4090 GPU (x-axis) and quality measured using sketch-and-ridge regression residual (y-axis).
    \textbf{Lower is better for both axes.}}
  \label{fig:sketch_and_ridge_pareto}

\end{figure}
We present a table of aggregated speedups over baselines across all RandNLA tasks, input shapes, and datasets in \Cref{tab:randnla_speedup_summary} (run on a RTX 4090).
\begin{table}[t]
  \centering
  \caption{Geomean speedups of \method vs baselines aggregated over shapes, datasets and configs. \textbf{Global geomean vs next best baseline: 1.73x.}}
  \label{tab:randnla_speedup_summary}
  \small
  \setlength{\tabcolsep}{4pt}
  \resizebox{\linewidth}{!}{%
  \begin{tabular}{ccccc}
    \toprule
    Task & \shortstack{SJLT\\(cuSPARSE)} & \shortstack{SJLT\\(GraSS Kernel)} & \shortstack{Subsampled\\FHT} & \shortstack{Dense Gaussian\\(cuBLAS)} \\
    \midrule
    \shortstack{Gram matrix\\approximation} & 2.67 & 4.17 & 16.22 & 7.64 \\
    \addlinespace[0.2em]
    \cmidrule(lr){1-5}
    OSE & 2.69 & 4.16 & 16.20 & 7.63 \\
    \addlinespace[0.2em]
    \cmidrule(lr){1-5}
    \shortstack{Sketch-and-\\ridge regression} & 1.42 & 1.54 & 3.94 & 3.10 \\
    \addlinespace[0.2em]
    \cmidrule(lr){1-5}
    \shortstack{Sketch+\\Solve} & 1.37 & 1.41 & 3.33 & 2.34 \\
    \bottomrule
  \end{tabular}}
\end{table}

Corresponding detailed plots and ablations appear in \Cref{sec:app_additional_experiments}.

\subsection{End-to-end ML: GraSS for Data Attribution}
We integrate \method into GraSS~\cite{hu2025grass}, a scalable data attribution pipeline built around compressing and storing per-example training gradients.
At a high level, GraSS constructs a \emph{feature cache} for the training set by compressing per-example gradients using a sparsification followed by a random projection (or sketch).
In the attribution phase, GraSS computes an analogous representation for a test/query example and produces an attribution vector using the cached features and a small downstream solve.
In GraSS, the random projection step is a key bottleneck: it is invoked for every training example during cache construction (and for every query at inference time), so kernel efficiency directly impacts end-to-end wall time.

We replace GraSS's sparse projection (its SJLT CUDA kernel) with \method while keeping all other pipeline components fixed.
We adapt the code from their open-source repository directly~\cite{grass_github}.
We focus our evaluation on the time spent in the projection/sketching step during feature cache construction.

\paragraph{Attribution quality via the linear datamodeling score (LDS).}
We evaluate attribution quality using the \emph{linear datamodeling score} (LDS) introduced in TRAK~\cite{park2023trak}, which GraSS also adopts.
LDS is a counterfactual metric: it measures how well an attribution method predicts how the model's output on a fixed example $z$ would change if we retrained on different subsets of the training set.
\textbf{Note that higher LDS is better.}

\Cref{fig:grass_mnist_pareto} shows the quality--speed Pareto frontier for end-to-end GraSS on an MLP trained on MNIST~\cite{lecun2002gradient}.
We refer the reader to \Cref{sec:app_additional_experiments_grass} and \cite{hu2025grass} for full details on the GraSS pipeline and the LDS metric.

\begin{figure}[t]
  \centering
  \includegraphics[width=\linewidth]{figures/fig_grass_camera_ready_mlp_mnist_pareto.pdf}
  \includegraphics[width=\linewidth]{figures/fig_grass_camera_ready_mlp_mnist_proj_only_bar.pdf}
  \caption{\textbf{End-to-end GraSS evaluation on MNIST.}
    We compare sketching methods by sketch runtime per sample on an NVIDIA RTX A6000 GPU (x-axis), and quality measured using the LDS metric (y-axis).
    \textbf{Lower is better for time (x-axis), higher is better for LDS (y-axis).}
    \method pushes the Pareto frontier and achieves up to $\sim 3.2\times$ speedup over the GraSS baseline while maintaining attribution quality (top).
  }
  \label{fig:grass_mnist_pareto}

\end{figure}
