% Macros for the FlashSketch paper
% NOTE: Keep TODOs minimal in the main body; use sparingly.
\newcommand{\TODO}[1]{\textcolor{red}{\textbf{TODO:} #1}}
\newcommand{\verbinline}{\Verb}
\DefineVerbatimEnvironment{Code}{Verbatim}{fontsize=\small}
\newcommand{\OrangeSectionStart}{\begingroup\color{orange}}
\newcommand{\OrangeSectionEnd}{\endgroup}

% Naming.
% Convention used throughout:
%   - \sketchfamily is the *distribution* (BlockPerm-SJLT).
%   - \method is the *kernel / implementation* (FlashSketch).
% Use \xspace so punctuation like "\method," does not render as "FlashSketch ,".
\newcommand{\method}{\textsc{FlashSketch}\xspace}
\newcommand{\sketchfamily}{\textsc{BlockPerm-SJLT}\xspace}

% Fast baseline: block-row sampling.
% We use the same name for the distribution and its kernel, since the
% implementation is essentially "the method".
\newcommand{\blockrowsketch}{\textsc{FlashBlockRow}\xspace}

% Paper-facing labels (consistent legend strings)
\newcommand{\labelFlash}{\method}
\newcommand{\labelGrass}{\textsc{GraSS-SJLT} (baseline)}
\newcommand{\labelBlockRow}{\blockrowsketch{} (fast, fragile)}

% Notation
\newcommand{\R}{\mathbb{R}}
\newcommand{\E}{\mathbb{E}}
\newcommand{\Prb}{\mathbb{P}}
\newcommand{\norm}[1]{\left\lVert #1\right\rVert}
\newcommand{\ip}[2]{\left\langle #1, #2\right\rangle}
\newcommand{\bigO}{\mathcal{O}}

% Helpers
\newcommand{\defeq}{\vcentcolon=}

% Common dimensions
\newcommand{\din}{d}
\newcommand{\dout}{k}
\newcommand{\npts}{n}

% Kernel params
\newcommand{\Br}{\ensuremath{B_r}}
\newcommand{\Bc}{\ensuremath{B_c}}
\newcommand{\Tn}{\ensuremath{T_n}}
\newcommand{\Tk}{\ensuremath{T_k}}
\newcommand{\spar}{\ensuremath{s}}

% Numbers
\newcommand{\geomeanSpeedup}{1.7\xspace}
